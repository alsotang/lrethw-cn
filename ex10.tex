\chapter{Exercise 10: Frugality In All Things}

You actually now know everything you need to use very basic regular expressions.
You know how to match all of the following things:

\begin{description}
\item[characters] Do this by just typing the chars, and using \verb|\| to escape
    regex symbols you want to match.
\item[any char] Use the \verb|.| (dot) to match any one char.
\item[char sets] Use \verb|[]| to make a set, including ranges of characters
    to match like \verb|[0-9]|
\item[inverted sets] Put a \verb|^| inside a set and it inverts: \verb|[^a-z]|.
\item[beginning anchor] A \verb|^| at the beginning of a regex says to "anchor" it
    so that it only matches at the beginning.
\item[ending anchor] A \verb|$| at the end of a regex anchors to the end so it only
    matches at the end.  Combine with \verb|^| to match exactly.
\item[optional modifier] Put a \verb|?| after a regex symbol, char, or set and it
    will make that thing optionally matched.
\item[one-or-more] A \verb|+| after a regex symbol, char, or set and it will
    match one-or-more of them.
\item[zero-or-more] A \verb|*| after a regex symbol, char, or set and it's optionall
    there or repeated (zero or more).
\end{description}

Mostly everything after this is additional ways to do the following to the above
concepts:

\begin{enumerate}
\item Limit repetition in more complex ways.
\item Group expressions.
\item Alternate between one or more expressions.
\item Special regex engine modifications for more complex tasks.
\end{enumerate}

In this exercise you'll learn to limit repetition in different ways by
specifying the "greediness" of the expression.  To learn this we'll
try matching various dates in different formats, but try to be as exact
as possible.

\begin{code}{ex10.txt}
\begin{Verbatim}
<< d['code/ex10.txt'] >>
\end{Verbatim}
\end{code}

Once you have that corpus text written out, here's the regex:

\begin{code}{ex10.regex}
\begin{Verbatim}
<< d['code/ex10.regex'] >>
\end{Verbatim}
\end{code}

I'm using three new kinds of syntax that all do the same thing:

\begin{enumerate}
\item If you put \verb|{X}| in the same place you'd put a \verb|+| (after
        something to repeat) then it will make sure that it is repeated
        \verb|X| times.
\item If you put \verb|{X,Y}| then it will make sure it repeates at least
    X times, but not more than Y times.
\item If you put a \verb|?| after any of the repetition symbols it means
    "non-greedy".  This is unfortunate because \verb|?| also means
    "optional" but just remember that placing it after repetition means
    "non-greedy".
\end{enumerate}

The way to explain "non-greedy" is simply that most regex engines try to 
match repetition to the biggest part of the corpus text possible.  If you
run into situations where you're matching too much with repetition, then
you can use \verb|?| to tell it to match the \emph{smallest} repetition
possible.

I can then break down each of the symbols I used in this regex file:

\begin{description}
\item[ex10.regex:1] In this regex I use \verb|{X}| to limit the month to
    3 characters (\verb|[A-Z]{3}|), the day to 2 digits (\verb|[0-9]{2}|), and
    the year to 4 digits (\verb|[0-9]{4}|).
\item[ex10.regex:2] This one I'm using the \verb|+?| to match as many characters
    for the month as possible, but to make it the smallest match possible.  Remember
    the \verb|?| after the \verb|+| does this.  I also use \verb|{2,4}?| at the
    end to match 2 to 4 characters for the year, but the \verb|?| will make 
    it match the smallest.
\item[ex10.regex:3] Finally I'm mixing these up to match 2 or 4 elements of
    the date.
\end{description}



\section{What You Should See}

When you run this it should match each of the dates correctly:

\begin{code}{ex10 Output}
\begin{Verbatim}
<< d['code/ex10.regex|regetron']['ex10.txt'] >>
\end{Verbatim}
\end{code}


\section{Extra Credit}

\begin{enumerate}
\item Write each of these out in verbose form and make sure you get the comments
    after each part.
\item Write out some index cards with all of the symbols you've learned so far,
    including these new ones.  On the front write the symbol/expression, and 
    on the back write the phrase for it.  For example, looking at the list
    at the beginning of this exercise, I'd put \verb|[]| on one side, and "char set"
    on the other side.
\item Drill with these cards for 15 minutes before you go to bed.  Put the ones
    you keep getting wrong into a separate pile.  The next day, 
    take breaks while you work and drill the ones you keep getting wrong only.  Repeat this
    until you get all of them correct.  Doing this before you go to bed
    will make your brain try to figure it out while you sleep.  Drilling the
    ones you don't know will efficiently train you in what you don't know.
\end{enumerate}

\section{Portability Notes}

Some regex engines do not have the \verb|{X,Y}| or \verb|{X}| syntax. Other
regex engines that have it become horribly slow when you give exact matching
counts.  I've found that it's best to just not use this syntax unless you
absolutely need to, and instead to your data validation in your software
after matching.



\chapter{Exercise 4: Matching Sets Of Characters}

Imagine you wanted to match only lines that had vowels.  To test this out
we'll use a contrived conversation about Cthuhlu:

\begin{code}{ex4.txt}
\begin{Verbatim}
<< d['code/ex4.txt|dexy'] >>
\end{Verbatim}
\end{code}

There's two lines with vowels, and then what Cthuhlu says doesn't have vowels.
We'd like to just ignore what he says so let's make a regex script to do that:

\begin{code}{ex4.regex}
\begin{Verbatim}
<< d['code/ex4.regex|dexy'] >>
\end{Verbatim}
\end{code}

You can either type that into \file{ex4.regex} and run it, or just run
\verb|regetron ex4.txt| and then enter that into the shell.

What that one line does is create a "set" of characters you want to match.
It's kind of like saying "I want to match 'a' or 'e' or 'i' or 'o' or 'u' or 'y'."
Another way to say this is, "Match any line with any one of these chars: aeiouy".

There's a few other things you can do with sets, but run this and see what you
get before I continue.

\section{What You Should See}

\begin{code}{ex4.out}
\begin{Verbatim}
<< d['code/ex4.out|dexy'] >>
\end{Verbatim}
\end{code}

See how it removed the line in the middle that didn't have vowels (the
line Cthulhu said).  It did this because, after scanning each character,
it didn't find 1 that matched the set you specified.

\section{Ranges Of Characters}

This is really handy, but it will be tedious if you had to enter in an
entire alphabet or all the numbers when you wanted to match those.  For
this common case you can use a range of characters by putting a 
'-' (dash) between them.  For example, \verb|[a-zA-Z]| will match
all characters "a through z" or "A through Z" thus matching all upper
or lowercase characters.

\section{Extra Credit}

\begin{enumerate}
\item Write another regex that matches only lowercase characters.  Use a range
    for it.
\item Add some numbers to the corpus text, and then write a regex with the
    numbers in a range (like 0-9).
\item Run regetron and use !data to set your phone number.  Now write a regex
    that matches your phone number using the range sets.
\end{enumerate}

\section{Portability Notes}

It's unclear how some regex engines will treat some human languages and alphabets
when doing a range.  Double check your documentation to make sure it's even
possible.



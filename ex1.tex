\chapter{Exercise 1: Your First Regex}

Start up Regetron again and type this in:

\begin{code}{ex1}
\begin{Verbatim}
!data "The lazy dog\nsleeps in the yard."
.*lazy.*
[A-Z][a-z]*
[a-z]\.
\end{Verbatim}
\end{code}

Enter each of these lines as commands to regetron, and make sure you get
them right.

\section{What You Should See}

Compare what you get for output with what I have below and
make sure it's the same.

\begin{code}{ex1 output}
\begin{Verbatim}
> !data "The lazy dog\nsleeps in the yard."
> .*lazy.*
0000: The lazy dog
> [A-Z][a-z]*
0000: The lazy dog
> [a-z]\.
0001: sleeps in the yard.
\end{Verbatim}
\end{code}

I've skipped showing you my entire shell session for starting
Regetron here, and won't show it unless it matters to the exercise.
Just assume that I started Regetron and entered this text in at the
\verb|> | prompts, then it printed the response under it.

What Regetron does is take you type, convert it to a regular expression,
and then runs that regular expression on each line.  In the above
we set the data to these two lines:

\begin{Verbatim}
The lazy dog
sleeps in the yard.
\end{Verbatim}

If a regular expression matches a line, then it gets printed.  In the
above you can see the first line gets matched twice, and the second
only by the last regular expression.

\section{Extra Credit}

\begin{enumerate}
\item Try entering a word that is not in either of the lines so you can see
    that Regetron prints nothing.  Like "pizza". Doesn't matter what it is.
\item Give regetron a file of text with \verb|regetron myfile.txt| and then
    try these same expressions to see what you get.  Make sure you do \emph{not}
    enter the first \verb|!data| command.
\item Do the same thing with the \verb|!load| command.
\item Flip "match" mode on with \verb|!match| and try these expressions again.
    Can you explain what's going on?
\item If you know Python, go read Regetron's code.  It's pretty small.
\end{enumerate}


\chapter{Exercise 2: Finding Specific Strings}

\begin{note}
If you have Regetron 1.1 then please upgrade to the latest Regetron so that
this exercise will work.  Version 1.2 and greater has the features you need.
\end{note}

The simplest thing you can do with a regex is find specific words or characters.
To do that you just type the characters.  The caveat though is you can't use any
of the characters that regex uses for its own syntax.  I'll show you how to do that
later.  For now, let's just stick to letters.

In this exercise we'll also learn how to run Regetron like it's a script.
Create a file with this in it named \file{ex2.txt}:

\begin{code}{ex2.txt}
\begin{Verbatim}
<< d['code/ex2.txt'] >>
\end{Verbatim}
\end{code}

I'll be calling these input files the "corpus" so you know they're the
"thing a regex searches through".

Make another file named \file{ex2.regex} that will be a little
"regex script" you'll run against the one above:

\begin{code}{ex2.regex}
\begin{Verbatim}
<< d['code/ex2.regex'] >>
\end{Verbatim}
\end{code}

You'll then run these two by doing \verb|regetron ex2.txt ex2.regex| 
to produce the output and see if you got it right.

\section{What You Should See}

When you run the command you'll get an output that looks like you 
were typing all of this in manually:

\begin{code}{ex2 output}
\begin{Verbatim}
<< d['code/ex2.regex|regetron']['ex2.txt'] >>
\end{Verbatim}
\end{code}

We won't do this all the time, but there will be times when I'll have you write
out large regex in "verbose" mode (covered later) and it'll be much easier
for you if you can put the whole thing in a file.

Taking a look at this exercise you can see that if you put a word, or any character,
then regetron finds it. If the word isn't in the corpus then it's not found, as with
"cucumber".

\section{Extra Credit}

\begin{enumerate}
\item Take this script and run regetron like normal against the file, but type
    the script in manually.  To do this you'll run regetron with 
    \verb|regetron ex2.txt| thus not including the script at the end.
\item Augment the corpus and try other words to make sure you know what's happening.
\end{enumerate}


\chapter{Exercise 6: Translating Simple Regex}

One major problem with regular expressions is you need to sit down
and manually translate them to and from English to understand them.  Eventually
this translation happens in your head, but in the beginning you'll
want to do the work.  Even after you've mastered "regex as a language"
you'll still want to do this on large regular expressions.

To help you do this, there is a "verbose mode" to the Python regex system, and
Regetron will let you write these out.  Just hit enter at a prompt, and then
type in the regex.  When your done enter a blank line.  Here's how you would
rewrite Exercise 1 to use verbose mode:

\begin{code}{ex6.regex}
\begin{Verbatim}
<< d['code/ex6.regex'] >>
\end{Verbatim}
\end{code}

The first thing you should notice is I can use comments to describe what
each character does.  The comment is after the '\#' (octothorpe) character.
You can also see that I can indent, as I do on lines 13 and 16, which helps
understand the structure.  I can do this because in verbose mode all
space and comments are ignored.  One more thing to understand is that to
make verbose mode work in regetron you have to do an \emph{empty} line
to start a block, and then another to end it, so there's \emph{two}
empty lines between blocks.

\section{What You Should See}

When you run this you should get something like this:

\begin{code}{ex6 Output}
\begin{Verbatim}
<< d['code/ex6.regex|regetron']['ex6.txt'] >>
\end{Verbatim}
\end{code}

Keep in mind that this is a technique we'll use for manually learning the
Regex language from here on, and for when you write complex regular
expressions that you have to maintain.  For simple regex like
this you would just write them out normally.

\section{Extra Credit}

This extra credit is simple but tedious  Write out \emph{all} of the regex you've
done so far (or as many as you can stomach) in verbose form.  This will
be tedious and annoying, but the translation process will make you learn
these symbols well.


\section{Portability Notes}

Many regex engines do not have a verbose mode.  Keep that in mind when you
write them and check the documentation.

\chapter{Exercise 15: Grouping And Captures}

\begin{quote}
You will need the latest Regetron for this exercise to work.  Make sure you have
at least version 1.4.  Do \verb|pip install --upgrade regetron| to get the new
one.
\end{quote}

You can match lines, but what if you just want the parts of a line that match?  
In this exercise you'll learn how to create a regex that has a "grouping" in it,
which tells the regex engine to return only that part, and not the whole line.
You create a group by surrounding the part of the regex you want to extract
with parenthesis \verb|()| characters.

Here's a corpus text with two lines I'll play with:

\begin{code}{ex15.txt}
\begin{Verbatim}
<< d['code/ex15.txt'] >>
\end{Verbatim}
\end{code}

The first line is just some pairs of numbers and letters, and I want to 
get only the numbers.  The second line has an email address in it and I
want to extract the email address.  To do that here's 4 regex:

\begin{code}{ex15.regex}
\begin{Verbatim}
<< d['code/ex15.regex'] >>
\end{Verbatim}
\end{code}

Each of these lines first shows the regex without parenthesis, then
with parenthesis.  The version without will just print the line like
you have been experiencing.  The version \emph{with} parenthesis
will print only what's been matched as a list of items.

\section{What You Should See}

When you run this you'll see first the line get matched, then the 
groups that match with the grouped (parenthesis) version after that.

\begin{code}{ex15 Output}
\begin{Verbatim}
<< d['code/ex15.regex|regetron']['ex15.txt'] >>
\end{Verbatim}
\end{code}

It's fairly straight forward, and all you need to know is put parenthesis
around a chunk of regex and regetron will print out a list of what matched.

\section{Extra Credit}

\begin{enumerate}
\item How do you match something with parenthesis then?  You know this but go
    back and slash through until you figure it out.
\item Create an email address that makes the email regex fail, then improve the
    email regex until it works.
\item Use the \verb,|, (or, alternate) symbol to have numbers or letters captured.
    Hint if you put it between two groups it will alternate between the groups.
\item How do you interpret these results now that it's doing an alternating grouping?
    Can you use just one group and still find numbers or letters?
\item Change the line so that it has something that's not letters or numbers
    and make sure your regex still only extracts numbers and letters.
\item Write an email regex that only captures the host name after the \verb|@| but
    still matches a regex. Hint: You don't have to put the parenthesis around the
    whole regex.
\item Read the Python \file{re} module documentation about groups.
\end{enumerate}

\section{Portability Notes}

Some regex engines are really bad at efficiently gathering captures, or do it
in weird ways.  Consult your API to see what's possible.

\chapter*{Introduction: Giants In Tiny Pants}

Do you like word problems? You know those annoying "math" questions you had to 
study in class simply because the SAT used them.  They went something like this:

\begin{quote}
"A train leaving a station at 10:50pm from San Francisco is travelling at
40MPH.  Another train leaving Chicago at 12:00pm is going 60MPH.  At which
point will the two trains cross paths."
\end{quote}

Even today, after studying years of advanced mathematics, statistics, and 
computer science, I \emph{still} can't solve these things.  The reason is
they're an attempt to hide an actual equation from you with convoluted
English wording.  Rather than just give you a straight math problem, like
$ax + b = c$ they want you to "reverse engineer" the real equation from
this pile of ambiguity.  In an attempt to simplify mathematics these 
types of questions actually end up obfuscating the real symbolic language
behind them.

Imagine if \emph{all} mathematics was like this where every single problem you
encountered, no matter how simple or complex, was only written as English
because nobody understood the symbols.  This would drive you crazy, even if you
didn't know mathematics.  Without the symbols of mathematics we'd find
mathematics degenerate to one Jacques Derrida style "obscurantisme terroriste"
paper after another.  They'd be no better than Philosophers!

Thankfully, humans invented symbols to succinctly describe the things that
symbols are best at describing, and left human languages to describe the
rest.\footnote{Thankfully, the things that can be described with symbols is pretty
small or else nobody would be able to read.}  The symbolic languages act as
building blocks to construct complex structures in a short space that would
take entire books in English to describe.

The power of symbolic languages is that, once you learn them, they can
describe things more quickly and accurately than a human language can.

\section{Regular Expressions Are Character Algebra}

Regular expressions, or regex, is a symbolic language that describe how to
identify a sequence of integers (or, more practically, characters) as matching
a required set.  Compared to mathematics regular expressions are pathetically
tiny.  There's only a few symbols to learn, and they have exact specific meanings.
This makes them easy to learn, given that you actually put in the time to 
learn to read and write them.

Effectively a regular expression is the algebra equation to a programming
language's word problem.  With them you can describe things that would take
mountains of code if you were to write it by hand.  This is why they exist,
but being symbolic languages most people simply hate them because most
people are never taught how to learn a symbolic language.


\section{How To Learn A Symbolic Language}

When you read a line of code like this:

\begin{lstlisting}
name.match(/^[\da-z]+\d$/);
\end{lstlisting}

You probably actually see this:

\begin{lstlisting}
name.match(justabunchofrandomjunkallsmashedtogether);
\end{lstlisting}

Your eyes most likely hit that part inside the parenthesis and try to process it as if it were
\emph{one cohesive word}.  The truth is each \emph{character} inside the parenthesis is actually
like a whole word or even a phrase.  What trips you up is you're used to reading sentences where
there's spaces between whole words and periods that end cohesive thoughts.  When you try to 
read this, you have to stop and tease the this one "word" apart into what it's actually doing.

However, if you've been programming for a while, you're already processing and understanding
a symbolic language, just one that's a bit closer to a human language.  What you need to do
is take this tiny bit of symbolic literacy skill and turn it into a real skill at reading
symbols.  To do this you need to do what you do when learning any language: Translate it
to and from the language you're currently know the best.

Imagine if you could write your regular expressions like this:

\begin{lstlisting}
name.match(/
        ^    # from the beginning
        [    # a set of chars containing
          \d # any digit
          a  # the letter a
          -  # through
          z  # the letter z
        ]    # end class
        +    # one or more
        \d   # any digit
        $    # to the end
        /);
\end{lstlisting}

Now it makes a bit more sense because every symbol in the regex is explained in
English.  You can now say, "Match from the start a set of one or more digits or
a-z followed by any digit to the end of the string." You could even get more
succinct and say, "Match completely one more digits or lowercase letters
followed by a digit."

In this book I'm going to teach you to read regular expressions by showing you how
to convert them back and forth from English such that you can break down any that you
find.  You'll also learn good habits for writing regular expressions, and more importantly
when to \emph{not} use them.  They're handy tools, but like any tool they're for
a particular job.

\section{The Giants In Tiny Pants Problem}

The problem that all symbolic languages have is an expert in them, using time
and continuous evolution, can eventually describe everything as one gigantic
equation.  They start with something simple, and then as they tack on hack
after hack to handle more and more edge cases things get worse.  In regular
expressions you end up with the infamous infamous
\href{http://www.ex-parrot.com/pdw/Mail-RFC822-Address.html}{Mail::RFC822::Address}
monstrosity.  Here's just the top 6 lines of all \emph{82} lines in this
one regular expression:

\begin{lstlisting}
(?:(?:\r\n)?[ \t])*(?:(?:(?:[^()<>@,;:\\".\[\] \000-\031]+(?:(?:(?:\r\n)?[ \t]
)+|\Z|(?=[\["()<>@,;:\\".\[\]]))|"(?:[^\"\r\\]|\\.|(?:(?:\r\n)?[ \t]))*"(?:(?:
\r\n)?[ \t])*)(?:\.(?:(?:\r\n)?[ \t])*(?:[^()<>@,;:\\".\[\] \000-\031]+(?:(?:(
?:\r\n)?[ \t])+|\Z|(?=[\["()<>@,;:\\".\[\]]))|"(?:[^\"\r\\]|\\.|(?:(?:\r\n)?[ 
\t]))*"(?:(?:\r\n)?[ \t])*))*@(?:(?:\r\n)?[ \t])*(?:[^()<>@,;:\\".\[\] \000-\0
31]+(?:(?:(?:\r\n)?[ \t])+|\Z|(?=[\["()<>@,;:\\".\[\]]))|\[([^\[\]\r\\]|\\.)*\
](?:(?:\r\n)?[ \t])*)(?:\.(?:(?:\r\n)?[ \t])*(?:[^()<>@,;:\\".\[\] \000-\031]+
... for 82 lines ...
\end{lstlisting}

This isn't regular expression's fault, it's the fault of people abusing the
tool when there's other tools more suitable.  In this case, a simple lexer
lexer that used smaller regular expressions would work better and most likely
be faster and less error prone.  More importantly, a lexer could report
\emph{errors} and tell you where something failed to pass.  Tools like lex,
re2c, and Ragel all make this easier than the above.

I call this the "giant in tiny pants problem" because you're taking something
that's actually a giant complex piece of code, and trying to cram all of it
in the tiny pants of a regular expression.  In the above example they're trying
to cram an email address parser into one a regular expression, and what they
end up with is not a succinct clean expression, but still a giant bursting at
the seams.

\section{Avoiding Giants In Tiny Pants With Parsing}

The key to using regular expressions correctly is to know where their
usefulness ends and when you need to bust out a lexer.  You also need to know
where a lexer falls down and when a parser is the right tool.  When you
use regular expressions to simplify creating lexers that feed into simple
parsers you then have a set of tools for cleaning and accurately parsing
text without going insane.

In this book I'm going to subversively teach you parsing, but I'm going
to be very practical and straight forward about it.  No NFA to DFA conversion.
No crazy explanations of push down finite state automata.  Just practical
code that gets you introduced to the basics of parsing, understanding the
core theory, and then actually using them to get work done.


\section{How To Use This Book}

As with all of my "Learn The Hard Way" books, the best way to use this
book is to do the work.  That means sitting down with each exercise,
actually typing all of the code in, reading about what you did, and
doing the extra credit.  You cannot learn these concepts without
actually doing the work and practicing them.  Otherwise you might
as well just save your money and continue to suck at programming.

A good way to do the book is to go "low and slow".  Rather than
sit down on Saturday and cram 8 exercises out in a 10 hour marathon,
you should spend 1 hour a night and do 1 or 2 exercises.  Sometimes
you'll blaze through exercises and other times you'll get stuck and
need to study.  Some exercise you might have to struggle with for
a week before moving on.  But, your average should be about 1 a day
and consistent steady study is more important than cramming.

The reason I give this advice is you've got to train and retrain
your brain to understand this new language, and the best way to
do that is with repetition and practice.  Otherwise you just cheat
yourself and end up \emph{thinking} you've learned something when
you've really done nothing but become familiar with the material.

Enjoy the book, and if you need help, just email \href{mailto:help@shavian-publishing.com}{help@shavian-publishing.com}
and I'll give you pointers.

